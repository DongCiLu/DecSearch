\section{Related Works}
\label{relatedwork}

Existing work on shortest path/distance can be classified into exact approaches and approximate approaches. 

\textbf{Exact Approaches:} Majority of the exact approaches are based on either 2-hop cover \cite{Cohen:2002:RDQ:545381.545503}, \cite{Akiba:2013:FES:2463676.2465315} or tree decomposition \cite{Akiba:2012:SQC:2247596.2247614}, \cite{Wei:2010:TES:1807167.1807181}. For the former one, finding optimal 2-hop covers is a challenging problem. \cite{Akiba:2013:FES:2463676.2465315} takes a different approach that solving 2-hop cover problem with graph traversals which has better scalability. \cite{Jin:2012:HLA:2213836.2213887} borrowed the highway concept from shortest path algorithms on road networks and construct a spanning tree as a "highway". \cite{Fu:2013:IIB:2536336.2536346} introduced an effective disk-based label indexing method based on independent set.

\textbf{Approximate Approaches:} Since the exact approaches do not scale well, approximate algorithms are also well studied for large-scale complex networks. Landmark based algorithms are extensively studied for approximating shortest path/distance \cite{Thorup:2005:ADO:1044731.1044732}, \cite{Goldberg:2005:CSP:1070432.1070455}, \cite{Potamias:2009:FSP:1645953.1646063}, \cite{floreskul2014memory}, \cite{Maier:2011:INS:1993077.1993079}. Although theoretical study of such algorithms does not reveal promising results \cite{Thorup:2005:ADO:1044731.1044732}, they work well in practice. \cite{Potamias:2009:FSP:1645953.1646063}, \cite{6927522} studied various landmark selection strategies for constructing better indexes. A common problem for distance-only indexes is that they do not perform well for close pairs of vertices \cite{Akiba:2012:SQC:2247596.2247614}. Algorithms \cite{Gubichev:2010:FAE:1871437.1871503}, \cite{tretyakov2011fast}, \cite{6399472} which index shortest paths are proposed to alleviate this problem. Beside landmark based approaches, there are several other approximate approaches. \cite{7004250} forms the shortest path problem as a learning problem to predicting pairwise distance. \cite{Zhao:2010:OSP:1863190.1863199} maps vertices to low-dimension Euclidean coordinate spaces to answer distance queries in constant time.

\textbf{Combining online search with indexes:}
\cite{Goldberg:2005:CSP:1070432.1070455} uses A* search for online query based on indexes constructed by landmark based algorithms. However, the cost of each A* search is still very high for large scale networks. \cite{Gubichev:2010:FAE:1871437.1871503} perform BFS on a sub-graph generated by the labels of source and target vertex. Although search space is greatly reduced, it still needs around seconds to handle graphs with millions of vertices.
