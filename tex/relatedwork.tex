\section{Related Works}
\label{relatedwork} 

The majority of the exact approaches are based on either 2-hop cover~\cite{Cohen:2002:RDQ:545381.545503,Akiba:2013:FES:2463676.2465315} or tree decomposition~\cite{Akiba:2012:SQC:2247596.2247614,Wei:2010:TES:1807167.1807181}. For the former one, finding optimal 2-hop covers is a challenging problem. Reference~\cite{Akiba:2013:FES:2463676.2465315} takes a different approach that solve 2-hop cover problem with graph traversals which has better scalability. Reference~\cite{Jin:2012:HLA:2213836.2213887} borrowed the highway concept from shortest path algorithms on road networks and construct a spanning tree as a "`highway"'. Reference~\cite{Fu:2013:IIB:2536336.2536346} introduced an effective disk-based label indexing method based on independent set.

Since the exact approaches do not scale well, approximate algorithms are also well studied for large-scale complex networks. Landmark based algorithms are extensively studied for approximating shortest path/distance~\cite{Thorup:2005:ADO:1044731.1044732,Goldberg:2005:CSP:1070432.1070455,Potamias:2009:FSP:1645953.1646063,floreskul2014memory,Maier:2011:INS:1993077.1993079, das2010sketch}. Although theoretical study of such algorithms does not reveal promising results~\cite{Thorup:2005:ADO:1044731.1044732}, they work well in practice. Reference~\cite{Potamias:2009:FSP:1645953.1646063,6927522} studied various landmark selection strategies for constructing better indexes. A common problem for distance-only indexes is that they do not perform well for close pairs of vertices~\cite{Akiba:2012:SQC:2247596.2247614}. Algorithms~\cite{Gubichev:2010:FAE:1871437.1871503,tretyakov2011fast,6399472} that index shortest paths are proposed to alleviate this problem. The problem has also been formed as a learning problem ~\cite{7004250} and mapped to low-dimension Euclidean coordinate spaces~\cite{Zhao:2010:OSP:1863190.1863199} to find approximated answers.

Our work falls into the category of applying online searches to indexed graphs. A* search is used for online query based on indexes constructed by landmarks to answer exact shortest path queries~\cite{Goldberg:2005:CSP:1070432.1070455}. However, the cost of each A* search is still very high for large scale networks. There are also a few work~\cite{Gubichev:2010:FAE:1871437.1871503, 6399472} perform online searches on sub-graphs consist of labels of source and target vertex. Although search space in ~\cite{Gubichev:2010:FAE:1871437.1871503} is small, path accuracy and diversity are compromised due to the constraints of the search space. In~\cite{6399472}, by controlling the width of shortest path tree the online search visits, differentiated accuracy levels can be achieved. But the search space can only be coarsely controlled. For graphs with power-law degree distribution, it is impractical to have the width larger than $1$.
