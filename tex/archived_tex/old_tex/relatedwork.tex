\section{Related Work}
\label{relatedwork}

In this section, we describe related work in three parts: first, we briefly survey existing methods for calculating shortest paths. Second, we describe methods on embeddings for graphs. Finally, we describe the applications of this work.

\textbf{Existing Methods:} Existing work on calculating shortest distances can be classified into two categories: exact approaches and approximate approaches.  The exact approach such as the Dijkstra algorithm has a computing complexity of $O(n^2)$ in general, and therefore not scalable for all-pair shortest distance calculations in large graphs. Another algorithm by Floyd-Warshall leverages dynamic programming to solve the all-pairs shortest paths in $O(n^3)$. On the approximate algorithm side, recent  state of the art algorithms also tried to combine bidirectional Dijkstra with A* algorithms to prune the search space. However, such algorithms are still too slow for large-scale graphs. Another interesting trend of research is to provide estimations on the path lengths rather than finding the actual paths. Such works typically avoid any kind of online Dijkstra/BFS traversals, but are considerably different from our work.

Our work falls into the category of trying to find approximate paths through preprocessing. Our goal is to preprocess a graph so that point-to-point queries can be answered approximately and at the computational overhead proportional to the path itself, not to the size of the network.  Key to the performance of such algorithms is the quality of found paths, i.e., the length to the optimal path. Previous theoretical research tends to get much more relaxed bounds than ours, as our empirical studies show that bounds as low as $1.1$ can be obtained in real-world graphs. In a small-world network, such as the graph described in the evaluation section, the exact distance is only guaranteed to lie within the interval. Therefore, the generated path is at most one hop longer.

\textbf{Graph Embedding Methods:} Our work on using the tree structure is closely related to general embedding methods, which aim to transform graphs into alternative representations. By doing so, we can obtain significant speed-ups
by embedding nodes into a different space and using a more efficient distance functions. Several landmark based approaches have been proposed in the literature. For example, Kleinberg discussed the problem of approximating network distances in real networks using landmarks and their theoretical implications. Our techniques are unique in the sense that trees are used as opposed to coordinates only. On another topic, it has been widely started to perform routing in geographic networks. Our work is different since we focus on graphs that exhibit complex social network behavior.

\textbf{Applications:} 
