

% This is a small sample LaTeX input file (Version of 10 April 1994)
%
% Use this file as a model for making your own LaTeX input file.
% Everything to the right of a  %  is a remark to you and is ignored by LaTeX.

% The Local Guide tells how to run LaTeX.

% WARNING!  Do not type any of the following 10 characters except as directed:
%                &   $   #   %   _   {   }   ^   ~   \

%%%%%%%%%%%%%%%%%%%%%%% file typeinst.tex %%%%%%%%%%%%%%%%%%%%%%%%%
%
% This is the LaTeX source for the instructions to authors using
% the LaTeX document class 'llncs.cls' for contributions to
% the Lecture Notes in Computer Sciences series.
% http://www.springer.com/lncs       Springer Heidelberg 2006/05/04
%
% It may be used as a template for your own input - copy it
% to a new file with a new name and use it as the basis
% for your article.
%
% NB: the document class 'llncs' has its own and detailed documentation, see
% ftp://ftp.springer.de/data/pubftp/pub/tex/latex/llncs/latex2e/llncsdoc.pdf
%
%%%%%%%%%%%%%%%%%%%%%%%%%%%%%%%%%%%%%%%%%%%%%%%%%%%%%%%%%%%%%%%%%%%

\documentclass[conference]{IEEEtran}

%% INFOCOM 2011 addition:
\makeatletter
\def\ps@headings{%
\def\@oddhead{\mbox{}\scriptsize\rightmark \hfil \thepage}%
\def\@evenhead{\scriptsize\thepage \hfil \leftmark\mbox{}}%
\def\@oddfoot{}%
\def\@evenfoot{}}
\makeatother
\pagestyle{headings}
%\usepackage{latex8}
\usepackage{setspace}
\usepackage{times}
%\usepackage{kbordermatrix}% http://www.hss.caltech.edu/~kcb/LaTeX.shtml
\newcommand{\noindex}{\hspace*{-0.8em}}%
\usepackage{pgf}
\usepackage{tikz}
\usetikzlibrary{arrows,automata}
\usepackage[latin1]{inputenc}
\usetikzlibrary{arrows}
\usepackage{epsfig,xspace}
\usepackage{url}
\usepackage{verbatim}
\usepackage{amsmath}
\usepackage{listings}
\usepackage{verbatim}

\usepackage{graphicx}
\usepackage{epstopdf}
\usepackage{mathtools}
%\usepackage{fix2col}
\usepackage{multirow}
\usepackage{varwidth}
\usepackage{algorithm}
\usepackage{algorithm}
\usepackage{algpseudocode}
%\usepackage[dvips]{graphicx}
%\usepackage{mss}
%\usepackage{subfigure}
\usepackage{subfigure}
%\usepackage{nopageno}
%\usepackage{fix2col}

\IEEEoverridecommandlockouts
\lstset{
  basicstyle=\footnotesize,
  columns=flexible,
  morecomment=[s]{/*}{*/}}

\newtheorem{Theorem}{Theorem}[section]
\newtheorem{lemma}[Theorem]{Lemma}
\newtheorem{proposition}[Theorem]{Proposition}
\newtheorem{corollary}[Theorem]{Corollary}

\newcommand{\remove}[1]{ }
\newcommand{\eg}{\emph{e.g.}}
\newcommand{\ie}{\emph{i.e.}}

\newenvironment{proof}[1][Proof:]{\begin{trivlist}
\item[\hskip \labelsep {\bfseries #1}]}{\end{trivlist}}


\newenvironment{definition}[1][Definition]{\begin{trivlist}
\item[\hskip \labelsep {\bfseries #1}]}{\end{trivlist}}
\newenvironment{example}[1][Example]{\begin{trivlist}
\item[\hskip \labelsep {\bfseries #1}]}{\end{trivlist}}
\newenvironment{remark}[1][Remark]{\begin{trivlist}
\item[\hskip \labelsep {\bfseries #1}]}{\end{trivlist}}

\renewcommand{\algorithmicrequire}{\textbf{Input:}}
\renewcommand{\algorithmicensure}{\textbf{Output:}}

\newcommand{\qed}{\nobreak \ifvmode \relax \else
      \ifdim\lastskip<1.5em \hskip-\lastskip
      \hskip1.5em plus0em minus0.5em \fi \nobreak
      \vrule height0.6em width0.5em depth0.0em\fi}

\def\sqzhuge{\vspace{-14pt}}
\def\sqzsec{\vspace{-10pt}}
\def\sqzsmall{\vspace{-8pt}}
\def\sqztiny{\vspace{-4pt}}
\remove{
\usepackage{vmargin}
\setpapersize{USletter}
\setmarginsrb{2.54cm}{2.54cm}{2.54cm}{1.5cm}{0pt}{0mm}{0pt}{8mm}
\setcounter{secnumdepth}{3}
}
\newcommand\Mark[1]{\textsuperscript#1}

\begin{document}
\pagenumbering{gobble}

\title{Fast and Distributed Approximation of\\ Shortest Paths in Large-scale Complex Networks}

\vspace{-0.4in}
\maketitle


%\maketitle

%\thispagestyle{empty}
%\pagestyle{empty}
\vspace{-1in}
\begin{abstract}

Finding near-shortest path in a fast and distributed way for large complex networks are challenging. Existing work for approximate shortest path are neither too slow or require too much storage and time overhead for networks with millions of edges. In this paper, we study methods for finding approximate results, where we hope to generate fast and distributed algorithms for finding near-shortest path based on graph embeddings. Specifically, the algorithm pre-computes and stores a digest of graph topology in each node, so that any point to point path query can be answered by finding a routing procedure from the source to destination, approximately but with high accuracy. We evaluate our algorithm on networks with different kinds of topology, such as scale free networks, random networks and sensor networks.

\end{abstract}

%\IEEEpeerreviewmaketitle


\section{Introduction}
\label{intro}

Introduced by the famous experiment conducted by Stanley Milgram, the small-world phenomenon, that most vertices can be reached from each other by a small number of hops, is found to be an fundamental characteristic in most real world networks. Although this property assure us the existence of such short paths between vertices in small world networks, due to the ever increasing number of vertices, this seemingly straightforward operation has become challenging. 
For example, for graph derived from LiveJournal social networks with 4 million vertices and 50 million edges \cite{Backstrom:2006:GFL:1150402.1150412}, online BFS/Dijkstra traversals will take around a minute to finish on a single commercial computer, even for A* search which can significantly reduce the number of vertices visited, it still require to access an average of 20K vertices for each query \cite{Potamias:2009:FSP:1645953.1646063}. 

Various networks which are commonly used in modern information technologies to model real-world phenomenon, such as online social networks (e.g., LinkedIn or Facebook), biological networks, road networks, etc have millions of vertices and billions of edges that even a cluster of machines can hardly deal with them. 
For example, in Figure~\ref{fig:trend}, we plot the runtime of BFS running on a Amazon EC2 cluster with 10 machines for networks of various size. Networks are derived from Facebook, Twitter and LiveJournal respectively \cite{Mcauley:2014:DSC:2582178.2556612}, \cite{Backstrom:2006:GFL:1150402.1150412}. We can see that even for a cluster of machines, a single BFS traversal takes around 3.5 seconds, and the time required for the traversal increases quickly as the scale of the network increases.

\begin{figure}[t]
    \centering
    \includegraphics[width = 3in]{../figures/BFS_compare.pdf}
    \caption{This figure shows the runtime of a single BFS on a Amazon EC2 cluster with 10 machines for graphs of various size.}
    \label{fig:trend}
\end{figure}

In this review, we are focusing on one of the most common variants of the shortest path/distance problem, to find a point-to-point shortest path/distance in a graph. This operation serves as the building block for many other tasks, and has many applications. For example, a natural application for road network is providing driving directions \cite{Abraham:2011:HLA:2008623.2008645}. In social networks, such applications include social sensitive search \cite{Vieira:2007:ESR:1321440.1321520}, analyze influential people \cite{Kempe:2003:MSI:956750.956769}. Estimating minimum round trip time between hosts without direct measurement is another application in technology networks \cite{Tang:2003:VLI:948205.948223}.

Usually in these applications, point-to-point shortest path/distance problem will be solved repeatedly for different source/target vertex pairs. A straight forward solution to this problem is to apply online traversal like BFS/Dijkstra for each query. But clearly, this approach is very inefficient due to the scalability issue discussed above. Due to the nature of applications' needs, most point-to-point shortest path/distance algorithms will use preprocessing in order to speed up the following online searching operations. One obvious way for preprocessing is to perform all-pair shortest paths operation such as Floyd-Warshall and store the results for each pair of vertices. When a query arrives, only constance time is required to return the results. However, this approach is also unacceptable due to the $O(V^3)$ runtime and $O(V^2)$ storage space. To balance between preprocessing and online query has become a key challenge in this research area \cite{Potamias:2009:FSP:1645953.1646063}, \cite{tretyakov2011fast}, \cite{Akiba:2012:SQC:2247596.2247614}, \cite{6399472}, \cite{Jin:2012:HLA:2213836.2213887}.

The point-to-point shortest path/distance problem has already been well studied in road networks, distance queries can be answered in less than a microsecond for the complete USA road network \cite{Abraham:2011:HLA:2008623.2008645}. Most of the methods on road networks take advantage of the spatial and planar-like properties a road network has \cite{Gubichev:2010:FAE:1871437.1871503}. Especially, Abraham et al. discovered that several fastest distance computation algorithms actually rely on a graph measure called highway dimension. However, these existing algorithms for road networks can hardly be applied to complex networks, due to the fact that the structure of most complex networks is nowhere near planar, and they usually do not have low highway dimensions like road networks. Actually, the locality of real-world complex networks is very poor, since the number of edges crossing clusters of the network is very large \cite{Leskovec08communitystructure}.

To further increase the scalability to handle large-scale complex networks, approximate methods have also been studied. Instead of returning an exact shortest path/distance between query vertex pairs, a near optimal path/distance is returned. Compared to online traversals for finding exact shortest path in seconds, it usually takes milliseconds for an algorithm to return an approximated shortest path \cite{Gubichev:2010:FAE:1871437.1871503}, \cite{tretyakov2011fast}, and microseconds to approximate shortest path distance \cite{Akiba:2013:FES:2463676.2465315}.

\subsection{Problem Formulation}

Our problem formulation is as follows: we investigate how to find exact or approximate point-to-point shortest paths/distance between any two vertices in extremely large networks. By extremely large, we are concerned with networks of million vertices, but in practice, such networks may even contain hundreds of millions of vertices. We are focusing on a specific kind of networks which is very common in real-world networks called complex network. They usually do not exhibit the same properties as simple ones such as road networks. For example, it is well known that these networks have their vertices' degrees conforming to the power law, and that vertices have relatively short distances between themselves. In most cases, the locality of these networks are very poor.

\subsection{Contributions}

In this review, we provide a comprehensive state-of-the-art study in algorithms on point-to-point shortest path/distance problems in large-scale complex networks. Existing algorithms usually contain two phases, distance calculation and online searching. After preprocessing, algorithms can return exact or estimated shortest path distance of two given vertices that can be used in applications where the underlying path is not important. The later one return an actual exact or approximate shortest path, and is essential for several applications require path information. Usually, online searching needs distance estimates to avoid large number of visited vertices, which is the main benchmark of these kind of algorithms.

\subsection{Paper organization}

The remaining of this review is organized as follows. In Section~\ref{distance}, we present several fundamental preprocessing methods being used by most of the algorithms and describes algorithms that calculate exact or estimated shortest path distance for any given vertex pair. We discuss online searching algorithms that finding actual paths in Section~\ref{path}. We conclude the whole review in Section~\ref{conclusion}.

\section{Motivational Case Study}
\label{motivationcase}

In this section, we first present a motivational case study, showing that there are critical differences between planar graphs and social networks. In particular, we show the performance of an algorithm applied to both planar graphs and social network graphs.

\subsection{Preliminaries}
\label{problemformulation}

In our problem, we consider a graph $G = (V,E)$, which represents a graph of vertex set $V$ and edge set $E$. For a source node $s$ and destination node $t$, we are interested in finding a path $p$ in the graph, which is an ordered sequence of $n$ vertices, such that,

\[
p = (v_0, v_1, ..., v_{n-1})
\]

where $(v_i,v_{i+1}) \in E$. We can write that $|p| = n$ to denote that length of $p$ is $n$. Given a pair of source and destination nodes, there may be many paths connecting them, which we denote as $P$. We can then define the distance between $s$ and $t$ as:

\[
distance(s,t) = min_{p\in P(s,t)}|p|
\]


Next we define path approximation ratio for a possibly non-optimal path $q$ as:

\[
ratio(q) = \frac{|q|}{distance(s,t)}
\]

Note that for any path $q$, the ratio defined here must be at least $1$. If $ratio(q)$ is equal to $1$, this path must be an optimal path by definition.

Based on this definition, suppose that given an approximation scheme that returns a path for every pair of nodes $s$ and $t$ (denoted as $p(s,t)$), we can easily define the average approximation ratio of all node pairs for a graph $G = (V,E)$ with $n$ nodes as:

\[
averageratio(G) = \frac{1}{n(n-1)}\sum_{s\in V, t\in V} ratio(p(s,t))
\]

\begin{figure}[t]
  \centering
  \includegraphics[width = 3in]{../figures/topology.pdf}
  \caption{This figure shows three landmarks, $\{d_1, d_2,d_3\}$, and six additional vertices. The coordinates for each node are shown in the table on the right.}
  \label{fig:lcr}
  \vspace{-3mm}
\end{figure}


\begin{figure*}[t]
\minipage{0.3\textwidth}
  \includegraphics[width=0.7\linewidth]{../figures/same_distance.pdf}
  \caption{This figure shows an example of coordinate distance may not represent real distance of two nodes.}\label{fig:same_distance}
\endminipage\hfill
\minipage{0.33\textwidth}
  \includegraphics[width=\linewidth]{../figures/lcr_socialnetwork.pdf}
  \caption{This figure shows distribution of hop difference between path found for the sensor network.}\label{fig:lcr_socialnetwork}
\endminipage\hfill
\minipage{0.33\textwidth}
  \includegraphics[width=\linewidth]{../figures/lcr_sensornetwork.pdf}
  \caption{This figure shows distribution of hop difference between path found for a social network.}\label{fig:lcr_sensornetwork}
\endminipage
\vspace{-0.1in}
\end{figure*}

\subsection{Existing Graph Embedding Approaches}

Previous methods based on graph embeddings have been developed under different names, such as landmark-based routing, coordinate based routing, among others, for planar graphs such as wireless sensor networks. To find a path between $s$ and $t$, this algorithm takes a two-staged approach: a pre-computation step that assigns each node with a coordinate vector, which is composed of distances from all vertices to a few selected landmark nodes, and a routing step that uses this pre-computed data to find paths between nodes with a complexity that is only proportional to the path distance. The overall procedure is shown in Figure~\ref{fig:lcr}, where a configuration of three landmarks and six vertices are illustrated.

The precomputation step, a set of landmarks is first chosen in the graph, based on the relative distances of nodes. Recent studies have pointed out that choosing the best set of landmarks for graph embeddings as NP-Hard, therefore, we assume they are chosen based on simple heuristics such as degrees of nodes. Next, for every node that is not a landmark, we compute the shortest path that connects each landmark to this node, by using the breadth-first search method. For a total of $m$ landmarks, there will be a total of $m$ breadth-first search procedures.

To perform routing and find paths, the Euclidean distance of the vectors are used to find the next hop based on the pre-computed coordinates. Note that here, the coordinates only contain distance information. Usually this works pretty well with relatively simple topologies. The key advantage is that the storage overhead of maintaining coordinate vectors is only proportional to the number of landmarks, which is usually sufficiently small compared to the total number of nodes.

To measure the quality of returned paths, we calculate the average approximation ratio of this figure. This is done by performing routing for every two pairs of nodes. For example, from $v_1$ to $v_3$, we can find that the distance between them $d(v_1,v_3) = \sqrt{5}$. Among the neighbors of $v_1$, $v_2$ has the shortest distance to $v_3$ as $1$. Hence, $v_2$ is chosen as the next hop. Therefore, the path found is $v_1,v_2,v_3$, which is optimal. One can verify that for this graph, the average approximation ratio is precisely $1$, meaning that all paths found are optimal.

\subsection{Comparison of Graph Embedding Performance on Complex Network vs Network with simple topology}



Although the coordinate based algorithm works well for networks with a simple topology, we find that it has poor performance for graphs with more complex topologies, such as social networks where a large range of degree distributions and long-distance edges dominate. The root cause is that the algorithm only encodes hop counts to a few landmarks, therefore, in many cases, the coordinates can no longer correctly represent the real distances of two nodes. Fig.~\ref{fig:same_distance} illustrates an example. To find a path from $v_2$ to $v_6$, using a greedy search will encounter problems between $v_5$ and $v_1$. In fact, both $distance(v_5,v_6)$ and $distance(v_1,v_6)$ are $2$, but intuitively, $v_1$ clearly has a shorter distance to $v_6$.

We next show a numerical study to illustrate the performance difference of coordinates applied to simple topologies and complex topologies. The first network is a social network which has power law degree distribution and relatively short diameter. The other network is a randomly generated network which has a topology similar to sensor networks. Each node has similar degree, following by a normal distribution. There are no long distance edges, nodes are only connected to nearby neighbors, so this network has a relatively long diameter. Both nodes have around $4,000$ nodes and around $176,000$ edges. Due the size of the graph, there are approximately $4000 * 4000$ different ordered node pairs. We only randomly choose $10,000$ pairs out of them and using coordinate routing to find path between each pair. Since the average shortest path of this two graph are different (approximate average length of shortest path for the social network is 3, while it is 40 for the generated random sensor network), instead of path approximation ratio, we show the distribution of length of path found by graph embedding approach minus shortest path length in Fig.~\ref{fig:lcr_socialnetwork} and Fig.~\ref{fig:lcr_sensornetwork}. From these two figure, we can see that compared to the network with simple topology, coordinate routing algorithm really performs poorly on complex networks. Actually. the average approximation ratio of randomly chosen $10,000$ pairs for the social network is $37.983$ compared to $1.014$ for random sensor network.
\begin{figure*}[t]
  \centering
  \includegraphics[width =6.5in]{../figures/tree.pdf}
  \caption{This figure shows forming a tree structure and then assigning coordinates to nodes based on their relative orderings in the tree.}
  \label{fig:tree}
  \vspace{-3mm}
\end{figure*}

\section{Algorithm Design}
\label{design}

In this section, we describe the design of our algorithm. We first present the overall algorithm. Then we describe a few optimizations to improve its performance.

\subsection{Algorithm Overview}
\label{architecture}

Our algorithm design is inspired by the previous work in that it is also built on top of a landmark-based framework. Specifically, it consists of the precomputation stage and the path-finding stage. We next describe these two stages separately.

\subsubsection{Precomputation}

The precomputation step involves choosing a few nodes as landmarks, and then computing for every node a ``topology digest'' based on this landmark. These digests will be used in the approximate path finding step later. The preprocessing, illustrated in Figure~\ref{fig:tree}, works as follows:

\textbf{Root Selection:} In this phase, we select a few nodes as roots, denoted as $t_0$, $t_1$, etc. These nodes will be used to construct trees. Usually, we select roots based on their degrees: those nodes that have the highest degrees will be chosen as the roots. On Figure~\ref{fig:tree}, however, we intentionally select the root as $t$ for illustration purpose, even though it does not have the highest degree. Specifically, Figure~\ref{fig:tree}(a) shows that a tree is constructed with the node $t$ as the root to all other nodes in the graph.

\textbf{Coordinate Assignments:} In this phase, we perform a BFS operation from the root to every other nodes in the tree structure. Therefore, for each node in the tree, we can find its children nodes, based on which we can assign them with coordinates. Figure~\ref{fig:tree}(b) to Figure~\ref{fig:tree}(d) show the children node coordinates. Specifically, one node assigns coordinates to its children nodes as follows: suppose this node's current coordinate is $(x)$ and it has $k$ children nodes. Each children node is assigned with $(x, 0)$, $(x,1)$, up to $(x,k-1)$. Note that such assignments have to propagate from level to level: first at the root, then propagates until the farthest leaves are reached. The coordinates assigned to each node is shown in Figure~\ref{fig:tree}(e).


\subsubsection{Shortest Path Approximation}

In this stage, the algorithm generates approximations for shortest paths between a pair of nodes, $s$ and $t$. The goal is to get the a small error, that is, the number of hops should be similar to the optimal path. Based on the pre-computed coordinates, the algorithm performs the following steps to generate such a path:

\begin{enumerate}
  \item Suppose that the path digests for node $s$ and $t$ are vectors $<a>$ and $<b>$, respectively, calculate their common ancestor $<c>$.
  \item Starting from the source node $s$, select the next hop as its parent until the last node of the common ancestor $<c>$ is reached.
  \item Starting from the last node of $<c>$, select the next hop using $<b>$'s coordinates until the destination node $t$ is reached.
  \item Return the path found in step 3.
\end{enumerate}

Clearly, for many cases, such a path is not optimal: it will have to go back to a root before making progress to the destination node. For example, in Figure~\ref{fig:tree}, we observe that from node $v_4$ to node $v_6$, a total of $4$ hops is taken ($v_4$,$v_1$,$t$,$v_3$,$v_6$), as opposed to the optimal path ($v_4$,$v_5$,$v_6$), which is $2$ hops. Next we describe a few optimizations to dramatically improve the performance of this approach.

\subsection{Design Optimizations}

In this section we explain our improvements to the original algorithm to obtain better approximations. As a first
step, we try to combine it with the coordinate based approach to find existing shortcuts within the paths. In the second step, we develop an ensemble of trees, and develop a hybrid method to calculate the distance, hence reducing the total path lengths. These improvements provide considerable
improvements in terms of the approximation quality, as
we will show in the experimental evaluation.

\begin{figure}[t]
  \centering
  \includegraphics[width = 2.4in]{../figures/topology2.pdf}
  \caption{This figure shows how to use shortcuts to reduce the length of found paths.}
  \label{fig:shortcut}
  \vspace{-3mm}
\end{figure}


\subsubsection{Shortcutting through Greedy Search on Weighted Coordinates}

We first describe the improvement called path shortcutting. The idea is that we try to go through the neighbors that directly lead to the destination node instead of going back to the root of the tree structures. The key idea is shown in Figure~\ref{fig:shortcut}, where we hope to obtain a path from $s$ to $t$. Using the tree structure alone, we will end up with a path on the left through the root. On the other hand, a shorter path exists along the ``leaves'' of the tree through $u$. Therefore, our goal is to search for such paths as shortcuttings, and return them whenever possible.

\begin{figure}[t]
  \centering
  \includegraphics[width = 3in]{../figures/topology3.pdf}
  \caption{This figure shows how weighted coordinates avoid local maximum along the path.}
  \label{fig:weighted}
  \vspace{-3mm}
\end{figure}

The basis of this design is that we use a modified version of the graph embedding based coordinates, called ``weighted coordinates'' as the key building block. The reason that we do not choose the original design of coordinates is that usually, coordinates do not reflect the real topologies in a perfect manner. An example of this is observed in Figure~\ref{fig:weighted}, which shows a case study where we would like to find a path from $v_9$ to $v_3$, illustrated with the orange line. The distances of each node to the destination are shown on the right half of this figure. Note that for the first hop from $v_9$ to $v_6$, the distance to the destination actually \emph{increases}, i.e., there is a local maximum reached at this hop. One solution to solve this problem is by observing that each dimension of the coordinate vectors should not be treated equally to contribute to the distance calculation. Instead, those coordinates that change \emph{the most} from the source to the destination should be given a larger weight, because they provide more information on the relative changes of path distances. In Figure~\ref{fig:weighted}, for example, the first dimension of the coordinate vectors changes from $5$ to $0$, which is the largest. Therefore, it should be given the highest weight.

Mathematically, we assign weights to dimensions based on the delta of the source and destination coordinates. Suppose that the source node $s$ has a vector of $(s_0, s_1, ..., s_{k-1})$, and the destination node $t$ has a vector of $(t_0, t_1, ..., t_{k-1})$, the weight coefficients $w_i$ ($i \in [0,k-1]$) are calculated as follows:

\[
w_i = \frac{|s_i-t_i|}{\sum_{j=0}^{k-1} |s_j-t_j|}
\]

Note that weight coefficients are only computed once at the source, and do not change along the path. The distance between an intermediate node $u$ to $t$ is then calculated as:

\[
d(u,t) = \sqrt{\sum_{j=0}^{k-1} w_j \times(u_j-t_j)^2}
\]


The results for the path in Figure~\ref{fig:weighted} are plotted as comparison. Observe that the local maximum no longer exists in the path.


\subsubsection{Tree Ensembles}

In this section we describe our second optimization to improve the performance, by using multiple trees instead of a single tree. We call this approach ``tree ensembles''. Recall that each tree is a collection of paths that connects the root with every other node in the graph without loops. Our improvement then constructs multiple trees, such that every node is assigned with more than one set of encodings. To build these trees, the algorithm starts breadth-first searches from different roots. To avoid reusing the same edges as much as possible, different BFS procedures will follow a different order of inserting nodes into the BFS search queue, so that those nodes that are used in one tree will be visited in a different order in a second tree.

After multiple trees are constructed, each node then has more than one encodings to use to lead to the destination nodes. In the algorithm, it will always pick the nearest one with the shortest distance such that only one tree is used.


\subsubsection{Putting Things Together}

We next describe how to put the two optimizations together. Specifically, every time a new path request is received, a packet delivery will switch between the ``greedy  mode'' and ``tree mode''. The key idea is that if the greedy mode is making smooth progress, we can try to find shortcuts with this approach. On the other hand, if local maximum areas are reached, the packet will now switch to the tree mode to route around the local maximum areas such that it will always be delivered to the destination. Therefore, the delivery ratio is always $100\%$ as long as the graph is not partitioned.  
\section{Distributed Implementations}
\label{implementation}

To handle extremely large graphs and large amount of queries, we implement our algorithm in distributed settings. Since decentralized search has low online search space complexity and does not have data dependency upon each other, it is well suitable to run in a parallel way. In this section, we discuss how to implement decentralized search on a distributed general graph processing platform, Powergraph\cite{180251}.

\begin{figure*}[ht]
    \centering
    \includegraphics[width=\linewidth]{./figures/new_illustrate/System.pdf}
    \caption{A system overview}
    \label{fig:system}
\end{figure*}

\subsection{Decentralized search vertex-program}

An overview of our shortest path query processing system is shown in \ref{fig:system}. The system first partition the graph with Powergraph onto multiple machines. Then several BFSs are performed to construct the index. After the index is built, multiple shortest path queries can run in parallel with decentralized search. Large volumes of queries can submit repeatedly, for which responses will be generated.

Decentralized search can be easily implemented as vertex-program in appliance to Powergraph's Gather-Apply-Scatter model. Each search instance contains approximated path and label of target vertex, as index is stored distributively and label of target vertex may not be accessible on each machine locally. In Gather phase, the search aim to find the neighbor with shortest LCA distance to the target. For each query, LCA distance $d_LCA$ calculated from labels $L$ of each neighbor and target vertex is collected and accumulated by finding the neighbor with the smallest LCA distance, A next hop candidate is returned after this phase. In apply phase, the program appends the candidate to the approximated path $p_{appr.}$ and and check whether the terminate criteria is met for each query. If so, the result path will be recorded and the query will be terminated. Otherwise, program will proceed to Scatter phase to start new vertex-programs on the candidate vertex and pass on the query instance to them. Algorithm \ref{alg:vc_dec} shows the detailed algorithm.

\begin{algorithm}
    \caption{Algorithm decentralized search vertex program running on $u$}
		\label{alg:vc_dec}
    \begin{algorithmic}
        \Function{gather}{$L(v)$, $L(t)$}
        %\Comment {Collect LCA distances from Nbrs}
        \State \Return $d_LCA(v, t)$, $v.id$
        \EndFunction

        \Function{sum}{$d_LCA(v_1,t)$, $vid1$, $d_LCA(v_2,t)$, $vid2$}
        %\Comment {Find Nbr with smallest LCA distance to target}
        \State \Return $min(d_LCA(v_1,t), d_LCA(v_2,t)$, related $vid$
        \EndFunction

        \Function{apply}{$p_{appr.}(s,t)$, $d_LCA(v,t)$}
        %\Comment {Update approximate path and check termination}
        \State $p_{appr.}(s,t) += vid$
        \If {terminate criteria is met}
            \State store $p_{appr.}(s,t)$
            \State $term$ = true
        \Else
            \State $term$ = false
        \EndIf
        \EndFunction

        \Function{scatter}{$p_{appr.}(s,t)$, $term$}
        %\Comment {activate candidates}
        \If {$\neg term$}
            \State Activate(v, $p_{appr.}(s,t)$)
        \EndIf
        \EndFunction
    \end{algorithmic}
\end{algorithm}

There are two thing to notice for the decentralized search vertex program. First, the search only update the approximated path $p_{appr.}$. There is no data dependency among multiple searches except when in the last results are stored to the same container for each machine. Second, the communication for decentralized search can happen during gather and scatter phase. In gather phase, the label of target vertex need to be passed to multiple machines, and the size is $O(k{\sigma}_{max})$. And each gather function return a $d_LCA$ along with its id that has $O(1)$ size. So for each query, only $(O(k{\sigma}_{max}) + O(1)) * M$ size of data is transferred where $M$ is the number of machines. In scatter phase, communication happens when a vertex is chosen as next hop and need to be activated, the whole search instance, including approximated path and label of target vertex need be transmitted to the new vertex program. For each query, $O(k{\sigma}_{max}) + O({\sigma}_{max})$ size of data may be transferred.

The low memory and communication cost, along side with light data dependency makes multiple decentralized search instance very suitable to run in parallel. To modify the vertex program for parallel processing, each vertex program maintain a list of search instance. During gather phase, label of target vertex for each query is transmitted to other machines, and $d_LCA$ is calculated for each query. Each query is updated during apply phase. In scatter phase, each query is examined for whether activating a certain vertex or not.

\subsection{Distributed tie breaking strategy}
If a tie happens, according to tie strategy, multiple neighbors may be returned as candidates at gather phase. Then at apply phase, the search instance will be fork itself into multiple search instance, each for a single candidate. The problem here is, not like centralized version, during the future computation, even a search instance find a shorter path than others, it cannot terminate other searches as such synchronization is quite costly in distributed settings. With this limitation, the search space are quite difficult to control and the search may end up with excessive number of child search instances. So in our implementation, at each step, only one candidate will be chosen as a "`main"' candidate. For candidates that are not "`main"' candidates, an extra terminate criteria will be applied. At the next step, if the search cannot find a shorter path than expected, i.e. $|p_{appr.}| + d_LCA$, then the search will stop, and no result will be recorded. This can control the search space effectively without compromise accuracy too much.

\subsection{Prune LCA computation}
It is possible to prune number of LCA computation required at each step for decentralized search. Suppose the search is visiting vertex $u$, which means the $d_LCA{u, t}$ has already been calculated in previous step. When traverse $u$'s neighbors, if a neighbor $v$ is a child on the indexed shortest path tree $T_i$, then the LCA computation for $v$ and $t$ on that tree $T_i$ does not need to be performed as it is certain that $d_LCA(i){v, t} > d_LCA(i){u, t}$. In practice, such way to pruning can reduce half of the total number of LCA computations of a single search on average.

\section{System Evaluation}
\label{evaluation}

\subsection{Evaluation Overview}
 
\subsection{Datasets}
 
\subsection{Methodology}
 
 
\subsection{Evaluation Results}
 
%\section{Application Case Study}
\label{casestudy}

\subsection{Problem Overview}
 
\subsection{Details}
 
\section{Related Work}
\label{relatedwork}

In this section, we describe related work in three parts: first, we briefly survey existing methods for calculating shortest paths. Second, we describe methods on embeddings for graphs. Finally, we describe the applications of this work.

\textbf{Existing Methods:} Existing work on calculating shortest distances can be classified into two categories: exact approaches and approximate approaches.  The exact approach such as the Dijkstra algorithm has a computing complexity of $O(n^2)$ in general, and therefore not scalable for all-pair shortest distance calculations in large graphs. Another algorithm by Floyd-Warshall leverages dynamic programming to solve the all-pairs shortest paths in $O(n^3)$. On the approximate algorithm side, recent  state of the art algorithms also tried to combine bidirectional Dijkstra with A* algorithms to prune the search space. However, such algorithms are still too slow for large-scale graphs. Another interesting trend of research is to provide estimations on the path lengths rather than finding the actual paths. Such works typically avoid any kind of online Dijkstra/BFS traversals, but are considerably different from our work.

Our work falls into the category of trying to find approximate paths through preprocessing. Our goal is to preprocess a graph so that point-to-point queries can be answered approximately and at the computational overhead proportional to the path itself, not to the size of the network.  Key to the performance of such algorithms is the quality of found paths, i.e., the length to the optimal path. Previous theoretical research tends to get much more relaxed bounds than ours, as our empirical studies show that bounds as low as $1.1$ can be obtained in real-world graphs. In a small-world network, such as the graph described in the evaluation section, the exact distance is only guaranteed to lie within the interval. Therefore, the generated path is at most one hop longer.

\textbf{Graph Embedding Methods:} Our work on using the tree structure is closely related to general embedding methods, which aim to transform graphs into alternative representations. By doing so, we can obtain significant speed-ups
by embedding nodes into a different space and using a more efficient distance functions. Several landmark based approaches have been proposed in the literature. For example, Kleinberg discussed the problem of approximating network distances in real networks using landmarks and their theoretical implications. Our techniques are unique in the sense that trees are used as opposed to coordinates only. On another topic, it has been widely started to perform routing in geographic networks. Our work is different since we focus on graphs that exhibit complex social network behavior.

\textbf{Applications:} 

\section{Conclusion}
\label{conclusion}

In this paper, we describe a novel method to combine online and offline processing to allow approximate shortest path searches for extremely large graphs with high distance accuracy, path diversity and low overhead. We demonstrate that different accuracy and overhead levels can be achieved by controlling the search space of decentralized searches. We also propose a more effective heuristic approach for constructing indexes of the network that can improve the accuracy of the decentralized search without increasing preprocessing and online searching overhead. We implement our algorithm for cloud computing graph processing platforms, and demonstrate that our system can handle extremely large graphs and processing millions of queries in parallel.


{\small
\bibliographystyle{IEEEtran}
\bibliography{reference}
}

\end{document}
