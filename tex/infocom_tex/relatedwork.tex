\section{Related Works}
\label{relatedwork} 

The majority of the exact approaches for shortest path problem are based on either 2-hop cover~\cite{Cohen:2002:RDQ:545381.545503,Akiba:2013:FES:2463676.2465315} or tree decomposition~\cite{Akiba:2012:SQC:2247596.2247614,Wei:2010:TES:1807167.1807181}. For the former one, finding optimal 2-hop covers is a challenging problem. Reference~\cite{Akiba:2013:FES:2463676.2465315} proposed to solve the 2-hop cover problem with graph traversals which achieves better scalability. 
%Variations of this work can also answer reachability quries~\cite{Yano:2013:FSR:2541167.2505724} and be applied to road networks~\cite{akiba2014fast} and dynamic networks~\cite{akiba2014dynamic}. 
Reference~\cite{Jin:2012:HLA:2213836.2213887} borrowed the highway concept from shortest path algorithms on road networks and constructed a spanning tree as a ``highway'' for complex networks. %Reference~\cite{Fu:2013:IIB:2536336.2536346} introduced an effective disk-based label indexing method based on independent sets. 
Most exact approaches do not scale well as the size of graphs increases.

Approximate algorithms have also been studied to achieve better scalability on large-scale complex networks. The majority of approximate algorithms is based on using landmarks as basis to construct offline indexes~\cite{Thorup:2005:ADO:1044731.1044732,Goldberg:2005:CSP:1070432.1070455,Potamias:2009:FSP:1645953.1646063,floreskul2014memory,Maier:2011:INS:1993077.1993079, das2010sketch, qi2013toward}. Although theoretical studies of such algorithms do not reveal promising results~\cite{Thorup:2005:ADO:1044731.1044732}, they work well in practice. Landmark selection strategies for indexing is a critical problem in landmark based algorithms. Such a problem is proven to be NP-hard and various heuristics are provided in~\cite{Potamias:2009:FSP:1645953.1646063,6927522}. ~\cite{qi2013toward} improved the index construction efficiency on distributed settings. Both shortest paths and distances can be indexed. A common problem for distance-only indexes is that they do not perform well for close pairs of vertices~\cite{Akiba:2012:SQC:2247596.2247614}. Algorithms that index shortest paths can alleviate this problem by exploit least common ancestors of close vertices~\cite{Gubichev:2010:FAE:1871437.1871503,tretyakov2011fast,6399472}. The problem has also been formulated as a learning problem ~\cite{7004250} and mapped to low-dimension Euclidean coordinate spaces~\cite{Zhao:2010:OSP:1863190.1863199} to find approximated answers. The landmark based approaches also extended to weighted graphs~\cite{yang2012finding}.

Our work falls into the category of applying online searches to indexed graphs. A* search is used for online query based on indexes constructed by landmarks to answer exact shortest path queries~\cite{Goldberg:2005:CSP:1070432.1070455}. Although it is able to answer exact shortest path queries, the cost of each A* search is still very high for large-scale networks. There are also a few works~\cite{Gubichev:2010:FAE:1871437.1871503, 6399472} that perform online searches on sub-graphs that consist of labels of the source and the target vertex for each query. Although the search space in ~\cite{Gubichev:2010:FAE:1871437.1871503} is small, path accuracy and diversity are compromised due to the constraints of the search space. In~\cite{6399472}, by adjusting the width of shortest path tree the online search visits, differentiated accuracy levels can be achieved. But for graphs with power-law degree distributions, it is impractical to expand the search to a width of more than $1$ as the search space will become too large.
