\section{Introduction}
\label{intro}

Various graphs are commonly used in modern information technologies to model real-world phenomenon, such as online social networks (e.g.,
LinkedIn or Facebook), biological networks, road networks, etc. Due to the ever increasing number of nodes, many seemingly straightforward operations have become challenging when extremely large graphs are involved. One problem that has drawn a lot of attention in the past is finding the shortest paths between any two nodes. This operation serves as the
building block for many other tasks, and has many applications. For example, in social networks, a person who wants to know a person of interest may want to find a shortest path of connections to reach this person, with only localized information visible to this user. For such applications, efficient and distributed methods are preferred than centralized algorithms.

Existing solutions to such problems, such as the well-known Floyd-Warshall algorithm, takes $O(V^3)$ time to find all-pair shortest paths. On the other hand, if we only need to find the shortest path between two nodes, a BFS (breadth-first-search) takes a worst-case time complexity of $O(|V|+|E|)$, where $|V|$ and $|E|$ stand for the number of vertices and edges, respectively. Later improvements to these early algorithms exist in the literature, however, they do not fully address our problem for the following two reasons: first, they do not easily scale to large graphs, which may comprise of millions or even billions of nodes such as the case of social network graphs. In such cases, finding solutions will be way too slow for most applications. Second, they are centralized, which means that they can not be easily implemented on distributed machines in an efficient manner. Finally, another way to achieve a trade-off between speed and space is to pre-compute and store the results for all pairs of nodes, and use them for quick lookups. This, however, is not feasible for large-scale networks with millions of nodes from the perspective of storage overhead.

Our key idea to solve this problem is to perform approximate, rather than accurate computations, where we hope to generate \emph{nearly optimal} paths. We observe that recently, a few algorithms have been developed to provide highly accurate estimates of the node distances. However, these approaches only give estimates of distances, but not the actual paths. In this work, our goal is to give the actual paths between nodes instead of their distance estimations.


\subsection{Problem Formulation}

Our problem formulation is as follows: we investigate how we can find paths that are nearly optimal between any two nodes in extremely large graphs. By extremely large, we are concerned with networks of at least several million nodes, but in practice, such graphs may even contain billions of nodes. The nature of the graphs can be flexible: they may not only include relatively simple graphs as exemplified in GIS systems, but also more complex networks such as social networks, which usually do not exhibit the same properties as simple ones. For example, it is well known that these networks have their nodes' degrees conforming to the power law, and that nodes have relatively short distances between themselves. Our goal is not only to give estimates on the hop counts, but also the actual paths in these networks.

\subsection{Contributions}

Our work is inspired by the previous work on graph embeddings found in the literature for simple graphs such as sensor networks, which in turn was based on the classical landmark based approaches. The overall idea of such frameworks is that they pre-compute and store a digest of graph topology in each node, so that any distance query can be answered by initiating a routing procedure from the the source node to the destination node. Previous results for graphs such as sensor networks have demonstrated that such routing leads to near-optimal paths with extremely high accuracy. We demonstrate in Section~\ref{motivationcase}, however, such approaches work very poorly for social network graphs. The reason is that social network exhibit much more complex topology connecting nodes, therefore, only encoding hop information as digests in graph embeddings is no longer sufficient. Based on this fact, we should also store the relative orderings of nodes in addition to their hop distances, so that the routing procedure can be made more effective. Based on the availability of such path-digests (as opposed to the distance-digests in graph embeddings), we then lightweight algorithms that can approximate shortest paths between any two nodes with a very small error. Additionally, we can also find a set of paths with the estimated shortest path distance with no additional overhead. In summary, contributions made in this paper are as follows:

\begin{enumerate}
  \item We introduce the concept of \emph{path-digests} for large graphs, which are topology-aware encodings as the computation basis. The path-digests are a set of orderings together with hop distances based on the constructed tree structures from a very small set of landmark nodes, computed as part of a graph preprocessing step.
  \item We develop a set of lightweight, yet highly effective, techniques that use path-digests to significantly improve the quality of shortest path finding, including several optimizations to improve path quality.
  \item We implement our proposed method in a fully functional large-scale graph processing engine called SNAP to find paths, where the procedure itself can be executed in parallel in a cluster of processing nodes.
  \item We evaluate our proposed algorithms against a number of large-scale, real-world datasets and graphs.
\end{enumerate}

Our proposed work has a wide range of applications, not only for social networks, but also for any types of networks where distance and path information need to be collected. It is also useful as a distributed routing approach for sensor networks and ad-hoc networks.

\subsection{Paper organization}

The remaining of this paper is organized as follows. In Section~\ref{motivationcase}, we present a motivational case study to illustrate that existing methods do not apply to complex networks such as social networks. Section~\ref{design} describes new algorithm that uses path digests generated by the precomputation step to calculate paths whose accuracy is better than the previous approaches. In Section~\ref{implementation}, we describe the implementation details of our algorithms. Section~\ref{evaluation} presents a systematic experimental evaluation that
shows the performance of our algorithms both in approximation quality and in overhead. Section~\ref{relatedwork} reviews related work. We conclude in Section~\ref{conclusion}. 
