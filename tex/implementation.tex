\section{Distributed Implementations}
\label{implementation}

To handle extremely large graphs and large amount of queries, we implement our algorithm in distributed settings. Since decentralized search has low online search space complexity and does not have data dependency upon each other, it is well suitable to run in a parallel way. In this section, we discuss how to implement decentralized search on a distributed general graph processing platform, Powergraph\cite{180251}.

\begin{figure*}[ht]
    \centering
    \includegraphics[width=\linewidth]{./figures/new_illustrate/System.pdf}
    \caption{A system overview}
    \label{fig:system}
\end{figure*}

\subsection{Decentralized search vertex-program}

An overview of our system is shown in \ref{fig:system}. The system first partition the graph with Powergraph onto multiple machines. Then several BFSs are performed to construct the index. After the index is built, multiple shortest path queries can run in parallel with decentralized search.

Decentralized search can be easily implemented as vertex-program in appliance to Powergraph's Gather-Apply-Scatter model. Each search instance contains approximated path and label of target vertex, as index is stored distributively and label of target vertex may not be accessible on each machine locally. In Gather phase, the search aim to find the neighbor with shortest LCA distance to the target. For each query, LCA distance $d_LCA$ calculated from labels $L$ of each neighbor and target vertex is collected and accumulated by finding the neighbor with the smallest LCA distance, A next hop candidate is returned after this phase. In apply phase, the program appends the candidate to the approximated path $p_{appr.}$ and and check whether the terminate criteria is met for each query. If so, the result path will be recorded and the query will be terminated. Otherwise, program will proceed to Scatter phase to start new vertex-programs on the candidate vertex and pass on the query instance to them. Algorithm \ref{alg:vc_dec} shows the detailed algorithm.
%Since the decentralized search is a state-less search itself, vertex data does not need to be updated during the Apply phase.

\begin{algorithm}
    \caption{Algorithm decentralized search vertex program running on $u$}
		\label{alg:vc_dec}
    \begin{algorithmic}
        \Function{gather}{$L(v)$, $L(t)$}
        %\Comment {Collect LCA distances from Nbrs}
        \State \Return $d_LCA(v, t)$, $v.id$
        \EndFunction

        \Function{sum}{$d_LCA(v_1,t)$, $vid1$, $d_LCA(v_2,t)$, $vid2$}
        %\Comment {Find Nbr with smallest LCA distance to target}
        \State \Return $min(d_LCA(v_1,t), d_LCA(v_2,t)$, related $vid$
        \EndFunction

        \Function{apply}{$p_{appr.}(s,t)$, $d_LCA(v,t)$}
        %\Comment {Update approximate path and check termination}
        \State $p_{appr.}(s,t) += vid$
        \If {terminate criteria is met}
            \State store $p_{appr.}(s,t)$
            \State $term$ = true
        \Else
            \State $term$ = false
        \EndIf
        \EndFunction

        \Function{scatter}{$p_{appr.}(s,t)$, $term$}
        %\Comment {Signal candidates}
        \If {$\neg term$}
            \State Activate(v, $p_{appr.}(s,t)$)
        \EndIf
        \EndFunction
    \end{algorithmic}
\end{algorithm}

There are two thing to notice for the decentralized search vertex program. First, the search only update the approximated path $p_{appr.}$. There is no data dependency among multiple searches except when in the last results are stored to the same container for each machine. Second, the communication for decentralized search can happen during gather and scatter phase. In gather phase, the label of target vertex need to be passed to multiple machines, and the size is $O(k{\sigma}_{max})$. And each gather function return a $d_LCA$ along with its id that has $O(1)$ size. So for each query, only $(O(k{\sigma}_{max}) + O(1)) * M$ size of data is transferred where $M$ is the number of machines. In scatter phase, communication happens when a vertex is chosen as next hop and need to be activated, the whole search instance, including approximated path and label of target vertex need be transmitted to the new vertex program. For each query, $O(k{\sigma}_{max}) + O({\sigma}_{max})$ size of data may be transferred.

The low memory and communication cost, along side with light data dependency makes multiple decentralized search instance very suitable to run in parallel. To modify the vertex program for parallel processing, each vertex program maintain a list of search instance. During gather phase, label of target vertex for each query is transmitted to other machines, and $d_LCA$ is calculated for each query. Each query is updated during apply phase. In scatter phase, each query is examined for whether activating a certain vertex or not.

\subsection{Distributed tie breaking strategy}
If a tie happens, according to tie strategy, multiple neighbors may be returned as candidates at gather phase. Then at apply phase, the search instance will be fork into multiple search instance, each for a single candidate. The problem here is, not like centralized version, during the future computation, even a search instance find a shorter path than others, it cannot terminate other searches as such synchronization is quite costly in distributed settings. With this limitation, the search space are quite difficult to control and the search may end up with excessive number of child search instances. So in our implementation, at each step, only one candidate will be chosen as a "main" candidate. For candidates that are not "main" candidates, an extra terminate criteria will be applied. At the next step, if the search cannot find a shorter path than expected, i.e. $|p_{appr.}| + d_LCA$, then the search will stop, and no result will be recorded. This can control the search space effectively without compromise accuracy too much.
